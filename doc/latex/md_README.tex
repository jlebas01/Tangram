A student project about the Tangram game made in C++

\subsection*{Getting started}

When you\textquotesingle{}re in the root directory of this project, follow the next steps \+: \subsubsection*{C\+Make}

First, If you have not did it already, you can build the game by executing the following command line \+: $>$\$ cmake ./cmake-\/build-\/debug \subsubsection*{Make}

Second, If you have not did it already, you can make the executable’s game by executing the following command line \+: $>$\$ cd cmake-\/build-\/debug \begin{quote}


$>$\$ make \end{quote}
\subsubsection*{Run}

Before using the run command line, you have to use the two aforementionned command in the right order. If you have already did it, you can run the game by executing the following command line \+: $>$\$ ./tangram

\subsection*{How to play}

Run the game with the following command line in the cmake-\/debug-\/build directory \+: $>$\$ ./tangram

You can play now.

\paragraph*{Launch Button}

You can create a new puzzle board if you click on the {\ttfamily Launch} button and use the following commands \+: $>${\ttfamily mouse click left} on a shape and drag to move it. \begin{quote}


$>${\ttfamily mouse click right} on a shape and drag to rotate it.

$>${\ttfamily press \textquotesingle{}Esc\textquotesingle{}} to exit this mode.

{\ttfamily press \textquotesingle{}s\textquotesingle{}} to save the current board as puzzle.

$>${\ttfamily press \textquotesingle{}d\textquotesingle{}} on a shape mouseovered to rotate it 45° anti clockwise.

$>${\ttfamily press \textquotesingle{}f} on a shape mouseovered to rotate it 45° clockwise.

$>${\ttfamily press \textquotesingle{}r\textquotesingle{}} to symmetrically reverse the shape.

$>$Note that last command rotates every shape to 180° except parallelogram which is \end{quote}
overturned (in a mirror fashion)

\paragraph*{Load Button}

If you click on the {\ttfamily Load} button, you can load a puzzle file and try to resolve it. You can use the following commands \+: $>${\ttfamily mouse click left} on a shape and drag to move it. \begin{quote}


$>${\ttfamily mouse click right} on a shape and drag to rotate it.

$>${\ttfamily press \textquotesingle{}Esc\textquotesingle{}} to exit this mode.

$>${\ttfamily press \textquotesingle{}d\textquotesingle{}} on a shape mouseovered to rotate it 45° anti clockwise.

$>${\ttfamily press \textquotesingle{}f} on a shape mouseovered to rotate it 45° clockwise.

$>${\ttfamily press \textquotesingle{}r\textquotesingle{}} to symmetrically reverse the shape.

$>$Note that last command rotates every shape to 180° except parallelogram which is \end{quote}
overturned (in a mirror fashion) \paragraph*{End Game}

The game will stop when you put the last shape at the right place. You will return to the main menu. When you solve a puzzle, the last shape dropped will be displayed in white and the game will freeze a for few seconds before you return to the main menu.

\subsection*{Documentation}

Here you can find H\+T\+ML files, La\+TeX files and P\+DF. \subsubsection*{H\+T\+ML}

Open with your browser ~\newline
 $>$\$ cd doc/html \begin{quote}


$>$index.\+html \end{quote}
\subsubsection*{La\+TeX}

$>$\$ cd doc/latex \subsubsection*{P\+DF}

Open with a P\+DF reader ~\newline
 $>$\$ cd doc/latex \begin{quote}


refman.\+pdf \end{quote}


\subsection*{Regenerate Documentation}

You can generate this document as needed. If you\textquotesingle{}re updating the code and the documentation, you should do execute in the root directory of this project \+: $>$\$ doxygen config-\/file

If you want customize the documentation generated, you could also configurate the following file \+: $>$\$ gedit config-\/file

\subsection*{Regenerate La\+TeX Documentation}

To generate the P\+DF documentation, execute the following commands \+: $>$\$ cd doc/latex \begin{quote}


$>$\$ make\end{quote}
